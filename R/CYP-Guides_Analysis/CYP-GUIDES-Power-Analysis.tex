% Options for packages loaded elsewhere
\PassOptionsToPackage{unicode}{hyperref}
\PassOptionsToPackage{hyphens}{url}
%
\documentclass[
]{article}
\usepackage{amsmath,amssymb}
\usepackage{iftex}
\ifPDFTeX
  \usepackage[T1]{fontenc}
  \usepackage[utf8]{inputenc}
  \usepackage{textcomp} % provide euro and other symbols
\else % if luatex or xetex
  \usepackage{unicode-math} % this also loads fontspec
  \defaultfontfeatures{Scale=MatchLowercase}
  \defaultfontfeatures[\rmfamily]{Ligatures=TeX,Scale=1}
\fi
\usepackage{lmodern}
\ifPDFTeX\else
  % xetex/luatex font selection
\fi
% Use upquote if available, for straight quotes in verbatim environments
\IfFileExists{upquote.sty}{\usepackage{upquote}}{}
\IfFileExists{microtype.sty}{% use microtype if available
  \usepackage[]{microtype}
  \UseMicrotypeSet[protrusion]{basicmath} % disable protrusion for tt fonts
}{}
\makeatletter
\@ifundefined{KOMAClassName}{% if non-KOMA class
  \IfFileExists{parskip.sty}{%
    \usepackage{parskip}
  }{% else
    \setlength{\parindent}{0pt}
    \setlength{\parskip}{6pt plus 2pt minus 1pt}}
}{% if KOMA class
  \KOMAoptions{parskip=half}}
\makeatother
\usepackage{xcolor}
\usepackage[margin=1in]{geometry}
\usepackage{graphicx}
\makeatletter
\def\maxwidth{\ifdim\Gin@nat@width>\linewidth\linewidth\else\Gin@nat@width\fi}
\def\maxheight{\ifdim\Gin@nat@height>\textheight\textheight\else\Gin@nat@height\fi}
\makeatother
% Scale images if necessary, so that they will not overflow the page
% margins by default, and it is still possible to overwrite the defaults
% using explicit options in \includegraphics[width, height, ...]{}
\setkeys{Gin}{width=\maxwidth,height=\maxheight,keepaspectratio}
% Set default figure placement to htbp
\makeatletter
\def\fps@figure{htbp}
\makeatother
\setlength{\emergencystretch}{3em} % prevent overfull lines
\providecommand{\tightlist}{%
  \setlength{\itemsep}{0pt}\setlength{\parskip}{0pt}}
\setcounter{secnumdepth}{-\maxdimen} % remove section numbering
\usepackage{multirow}
\usepackage{multicol}
\usepackage{colortbl}
\usepackage{hhline}
\newlength\Oldarrayrulewidth
\newlength\Oldtabcolsep
\usepackage{longtable}
\usepackage{array}
\usepackage{hyperref}
\usepackage{float}
\usepackage{wrapfig}
\ifLuaTeX
  \usepackage{selnolig}  % disable illegal ligatures
\fi
\IfFileExists{bookmark.sty}{\usepackage{bookmark}}{\usepackage{hyperref}}
\IfFileExists{xurl.sty}{\usepackage{xurl}}{} % add URL line breaks if available
\urlstyle{same}
\hypersetup{
  pdftitle={Power Analysis of CYP-Guides Experiment},
  hidelinks,
  pdfcreator={LaTeX via pandoc}}

\title{Power Analysis of CYP-Guides Experiment}
\author{}
\date{\vspace{-2.5em}}

\begin{document}
\maketitle

\hypertarget{power-analysis-of-cyp-guides}{%
\subsection{Power Analysis of
CYP-Guides}\label{power-analysis-of-cyp-guides}}

This is a continued analysis from my other
\href{https://github.com/kittysocks/portfolio/blob/main/R/CYP-GUIDES-Power-Analysis.docx}{.Rmd
document} that has the data description.

\hypertarget{t-test}{%
\subsection{T-Test}\label{t-test}}

\hypertarget{stratifying-and-randomly}{%
\subsection{Stratifying and Randomly}\label{stratifying-and-randomly}}

From my other .Rmd, I conducted a T-Test between the two. But I didn't
consider that the therapy type sample sizes were far apart in range.
There are 477 observations from Standard Therapy and 982 observations
from Genetically-guided Therapy. I could either perform a Wilcox T-Test
or stratify and randomly sample from the larger group. I'm going to
randomly sample and perform a new T-test.

Compared to doing a T-Test on the entire data (``0.515656957537065''),
we got a p-value of 0.9412888. This still is not statistically
significant. However, can we know that we did this p-value was found in
good faith? Did we have enough data to conclude our p-value was founded
as good?

\hypertarget{histogram-of-two-therapy-groups}{%
\subsection{Histogram of Two Therapy
Groups}\label{histogram-of-two-therapy-groups}}

Without the outliers

\includegraphics{CYP-GUIDES-Power-Analysis_files/figure-latex/unnamed-chunk-2-1.pdf}

I want to do a Power Analysis of the CYP-GUIDES data. I am currently
analyzing the LOS between the therapy two groups; the standard therapy
control group, and the genetically-guided therapy experimental group.

Power is the probability that a statistical test will show a significant
difference between the LOS of the two groups.

A Power Analysis will determine what sample size will ensure a high
probability that we can correctly see results for the next time this
experiment is performed.

If we use the sample size recommended by the Power Analysis, we will
know that regardless of the statistical testing performed, we can know
that we used enough data to make a good decision.

In order to do a Power Analysis, we want to know how much Power we want.
A common value is 0.8. That means we want an 80\% probability that we
will see statistical significance. If we ran the experiment 100 times,
80 of the experiments would show statistical significance. After
determining power, we need to determine the threshold for significance,
which is called alpha. A common value for alpha is 0.05. We then need to
determine the effect size, which is the difference between the two
observation's means and standard deviation. The effect size is the
overlap of data between the two groups. We can determine effect size by
using the below formula:

Effect size:
\(\frac{\text{The difference in means}}{\text{Pooled estimate standard deviations}}\)

Pooled estimate of standard deviations can be calculated using this:

\[
\sqrt{\frac{\text{s}^{2}+\text{s}^{2}}{\text{2}}}
\]

There are a lot of different ways to find effect size. It depends on
your data. Sometimes through literature, you might have to make an
educated guess. But luckily, we have the mean and the standard
deviations of the two groups from the experiment.

\global\setlength{\Oldarrayrulewidth}{\arrayrulewidth}

\global\setlength{\Oldtabcolsep}{\tabcolsep}

\setlength{\tabcolsep}{2pt}

\renewcommand*{\arraystretch}{1.5}



\providecommand{\ascline}[3]{\noalign{\global\arrayrulewidth #1}\arrayrulecolor[HTML]{#2}\cline{#3}}

\begin{longtable}[c]{|p{2.62in}|p{1.29in}|p{2.09in}}

\caption{Table\ 1.\ Descriptive\ statistics\ of\ the\ CYP-GUIDES\ study}\\

\ascline{0.75pt}{666666}{1-3}

\multicolumn{1}{!{\color[HTML]{666666}\vrule width 0.75pt}>{\raggedright}m{\dimexpr 2.62in+0\tabcolsep}}{\textcolor[HTML]{000000}{\fontsize{11}{11}\selectfont{\textbf{Therapy\ Type}}}} & \multicolumn{1}{!{\color[HTML]{666666}\vrule width 0.75pt}>{\centering}m{\dimexpr 1.29in+0\tabcolsep}}{\textcolor[HTML]{000000}{\fontsize{11}{11}\selectfont{\textbf{Mean\ LOS}}}} & \multicolumn{1}{!{\color[HTML]{666666}\vrule width 0.75pt}>{\centering}m{\dimexpr 2.09in+0\tabcolsep}!{\color[HTML]{666666}\vrule width 0.75pt}}{\textcolor[HTML]{000000}{\fontsize{11}{11}\selectfont{\textbf{Standard\ Deviation}}}} \\

\ascline{0.75pt}{666666}{1-3}\endfirsthead \caption[]{Table\ 1.\ Descriptive\ statistics\ of\ the\ CYP-GUIDES\ study}\\

\ascline{0.75pt}{666666}{1-3}

\multicolumn{1}{!{\color[HTML]{666666}\vrule width 0.75pt}>{\raggedright}m{\dimexpr 2.62in+0\tabcolsep}}{\textcolor[HTML]{000000}{\fontsize{11}{11}\selectfont{\textbf{Therapy\ Type}}}} & \multicolumn{1}{!{\color[HTML]{666666}\vrule width 0.75pt}>{\centering}m{\dimexpr 1.29in+0\tabcolsep}}{\textcolor[HTML]{000000}{\fontsize{11}{11}\selectfont{\textbf{Mean\ LOS}}}} & \multicolumn{1}{!{\color[HTML]{666666}\vrule width 0.75pt}>{\centering}m{\dimexpr 2.09in+0\tabcolsep}!{\color[HTML]{666666}\vrule width 0.75pt}}{\textcolor[HTML]{000000}{\fontsize{11}{11}\selectfont{\textbf{Standard\ Deviation}}}} \\

\ascline{0.75pt}{666666}{1-3}\endhead



\multicolumn{1}{!{\color[HTML]{666666}\vrule width 0.75pt}>{\raggedright}m{\dimexpr 2.62in+0\tabcolsep}}{\textcolor[HTML]{000000}{\fontsize{11}{11}\selectfont{Genetically-guided\ therapy}}} & \multicolumn{1}{!{\color[HTML]{666666}\vrule width 0.75pt}>{\centering}m{\dimexpr 1.29in+0\tabcolsep}}{\textcolor[HTML]{000000}{\fontsize{11}{11}\selectfont{173.4193}}} & \multicolumn{1}{!{\color[HTML]{666666}\vrule width 0.75pt}>{\centering}m{\dimexpr 2.09in+0\tabcolsep}!{\color[HTML]{666666}\vrule width 0.75pt}}{\textcolor[HTML]{000000}{\fontsize{11}{11}\selectfont{188.4277}}} \\

\ascline{0.75pt}{666666}{1-3}



\multicolumn{1}{!{\color[HTML]{666666}\vrule width 0.75pt}>{\raggedright}m{\dimexpr 2.62in+0\tabcolsep}}{\textcolor[HTML]{000000}{\fontsize{11}{11}\selectfont{Standard\ Therapy}}} & \multicolumn{1}{!{\color[HTML]{666666}\vrule width 0.75pt}>{\centering}m{\dimexpr 1.29in+0\tabcolsep}}{\textcolor[HTML]{000000}{\fontsize{11}{11}\selectfont{172.6080}}} & \multicolumn{1}{!{\color[HTML]{666666}\vrule width 0.75pt}>{\centering}m{\dimexpr 2.09in+0\tabcolsep}!{\color[HTML]{666666}\vrule width 0.75pt}}{\textcolor[HTML]{000000}{\fontsize{11}{11}\selectfont{149.4878}}} \\

\ascline{0.75pt}{666666}{1-3}



\end{longtable}



\arrayrulecolor[HTML]{000000}

\global\setlength{\arrayrulewidth}{\Oldarrayrulewidth}

\global\setlength{\tabcolsep}{\Oldtabcolsep}

\renewcommand*{\arraystretch}{1}

\end{document}
